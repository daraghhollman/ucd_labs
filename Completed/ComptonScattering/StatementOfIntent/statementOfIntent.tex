\documentclass[%
reprint,
amsmath,amssymb,
aps,
]{revtex4-2}

\begin{document}
	\title{PHYC30170 Physics with Astronomy and Space Science Lab 1;\\
	Compton Scattering; Statement of Intent}
	
	\author{Daragh Hollman}
	
	\date{\today}
	
	\maketitle
	
	\onecolumngrid
	\section{What is the aim of the experiment?}
		The aim of the experiment is to determine the energy of the scattered gamma ray and to measure the differential cross-section for Compton scattering both as a function of the scattering angle.
	
	\section{What measurements should be made and how?}
		The spectra of the gamma ray after scattering will be measured using a NaI(Tl) detector for angles between $20^\circ$ and $100^\circ$ in steps of $5^\circ$. The energy and net counts under the photopeak for each angle will be recorded.		
	
	\section{How will the final result be obtained from the experimental data?}
		The inverse of the energy of the gamma ray will be plotted against $1-\cos{\theta}$ and compared to the theory. The measured differential corss-section will be plotted against the scattering angle and compared to the theory.
		
	
	\section{What are the main safety concerns with the experiment and precautions that should be taken?}
		The main safety concerns with this experiment are pertaining to dealing with radioactive components and high voltage. We are working with $^{137}\text{Cs}$ which emits gamma rays and the photodiode requires high voltage electricity. Care is needed to be safe with both these elements.
	
\end{document}