\documentclass[%
reprint,
amsmath,amssymb,
aps,
]{revtex4-2}

\begin{document}
	\title{PHYC30170 Physics with Astronomy and Space Science Lab 1;\\
	Surface Plasmons; Statement of Intent}
	
	\author{Daragh Hollman}
	
	\date{\today}
	
	\maketitle
	
	\onecolumngrid
	\section{What is the aim of the experiment?}
		The aim of the experiment is to determine the angle of plasmon extinction and see how this depends on the wavelength of incident light and the thickness of the metal.
	
	\section{What measurements should be made and how?}
		Measurements will be made by varying the reflection angle using the stepper motors and recording the amount of light reflected in the detector. This will be repeated with different wavelengths of incident light and different thicknesses of metal.
	
	\section{How will the final result be obtained from the experimental data?}
		The intensity of the light reflected will be plotted against the angle. We expect to see a sharp decrease in this curve which the peak position will be recorded. This will be measured several times to calculate the angle of excitation. This will be repeated for varying sources of light and varying metal thicknesses.
	
	\section{What are the main safety concerns with the experiment and precautions that should be taken?}
		Precaution is needed when working with lasers and when working in the dark room. Do not look directly at the laser. Make sure the area surrounding the apparatus is clear of hazards such as bags and coats on the floor.
	
\end{document}